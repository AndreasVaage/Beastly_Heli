
\section{First flight}\label{first_flight}
Before we start on more advanced control theory we tested the helicopter and the given PD- and PID- controller for pitch and elevation respectively.
\begin{figure}[H] 
        \centering
        \setlength{\figureheight}{6cm}
        \setlength{\figurewidth}{10cm}
        % This file was created by matlab2tikz.
%
%The latest updates can be retrieved from
%  http://www.mathworks.com/matlabcentral/fileexchange/22022-matlab2tikz-matlab2tikz
%where you can also make suggestions and rate matlab2tikz.
%
\definecolor{mycolor1}{rgb}{0.00000,0.75000,0.75000}%
\definecolor{mycolor2}{rgb}{0.75000,0.00000,0.75000}%
\definecolor{mycolor3}{rgb}{0.75000,0.75000,0.00000}%
%
\begin{tikzpicture}

\begin{axis}[%
width=0.951\figurewidth,
height=0.419\figureheight,
at={(0\figurewidth,0.581\figureheight)},
scale only axis,
xmin=0,
xmax=14,
ymin=0,
ymax=25,
ylabel={p [deg]},
axis background/.style={fill=white},
legend style={legend cell align=left, align=left, draw=black}
]
\addplot [color=blue]
  table[row sep=crcr]{%
0	20\\
0.1	20\\
0.2	20\\
0.3	20\\
0.4	20\\
0.5	20\\
0.6	20\\
0.7	20\\
0.8	20\\
0.9	20\\
1	20\\
1.1	20\\
1.2	20\\
1.3	20\\
1.4	20\\
1.5	20\\
1.6	20\\
1.7	20\\
1.8	20\\
1.9	20\\
2	20\\
2.1	20\\
2.2	20\\
2.3	20\\
2.4	20\\
2.5	20\\
2.6	20\\
2.7	20\\
2.8	20\\
2.9	20\\
3	20\\
3.1	20\\
3.2	20\\
3.3	20\\
3.4	20\\
3.5	20\\
3.6	20\\
3.7	20\\
3.8	20\\
3.9	20\\
4	20\\
4.1	20\\
4.2	20\\
4.3	20\\
4.4	20\\
4.5	20\\
4.6	20\\
4.7	20\\
4.8	20\\
4.9	20\\
5	20\\
5.1	20\\
5.2	20\\
5.3	20\\
5.4	20\\
5.5	20\\
5.6	20\\
5.7	20\\
5.8	20\\
5.9	20\\
6	0\\
6.1	0\\
6.2	0\\
6.3	0\\
6.4	0\\
6.5	0\\
6.6	0\\
6.7	0\\
6.8	0\\
6.9	0\\
7	0\\
7.1	0\\
7.2	0\\
7.3	0\\
7.4	0\\
7.5	0\\
7.6	0\\
7.7	0\\
7.8	0\\
7.9	0\\
8	0\\
8.1	0\\
8.2	0\\
8.3	0\\
8.4	0\\
8.5	0\\
8.6	0\\
8.7	0\\
8.8	0\\
8.9	0\\
9	0\\
9.1	0\\
9.2	0\\
9.3	0\\
9.4	0\\
9.5	0\\
9.6	0\\
9.7	0\\
9.8	0\\
9.9	0\\
10	0\\
10.1	0\\
10.2	0\\
10.3	0\\
10.4	0\\
10.5	0\\
10.6	0\\
10.7	0\\
10.8	0\\
10.9	0\\
11	0\\
11.1	0\\
11.2	0\\
11.3	0\\
11.4	0\\
11.5	0\\
11.6	0\\
11.7	0\\
11.8	0\\
11.9	0\\
12	0\\
12.1	0\\
12.2	0\\
12.3	0\\
12.4	0\\
12.5	0\\
12.6	0\\
12.7	0\\
12.8	0\\
12.9	0\\
13	0\\
13.1	0\\
13.2	0\\
13.3	0\\
13.4	0\\
13.5	0\\
13.6	0\\
13.7	0\\
13.8	0\\
13.9	0\\
14	0\\
14.1	0\\
14.2	0\\
14.3	0\\
14.4	0\\
14.5	0\\
14.6	0\\
14.7	0\\
14.8	0\\
14.9	0\\
15	0\\
15.1	0\\
15.2	0\\
15.3	0\\
15.4	0\\
15.5	0\\
15.6	0\\
15.7	0\\
15.8	0\\
15.9	0\\
16	0\\
16.1	0\\
16.2	0\\
16.3	0\\
16.4	0\\
16.5	0\\
16.6	0\\
16.7	0\\
16.8	0\\
16.9	0\\
17	0\\
17.1	0\\
17.2	0\\
17.3	0\\
17.4	0\\
17.5	0\\
17.6	0\\
17.7	0\\
17.8	0\\
17.9	0\\
18	0\\
18.1	0\\
18.2	0\\
18.3	0\\
18.4	0\\
18.5	0\\
18.6	0\\
18.7	0\\
};
\addlegendentry{$\text{p}_{\text{ref}}$}

\addplot [color=black!50!green]
  table[row sep=crcr]{%
0	0\\
0.1	0\\
0.2	0.087890625\\
0.3	0.703125\\
0.4	2.021484375\\
0.5	4.04296875\\
0.6	6.240234375\\
0.7	8.0859375\\
0.8	9.580078125\\
0.9	10.72265625\\
1	11.689453125\\
1.1	12.568359375\\
1.2	13.359375\\
1.3	14.23828125\\
1.4	14.94140625\\
1.5	15.732421875\\
1.6	16.5234375\\
1.7	17.2265625\\
1.8	17.841796875\\
1.9	18.45703125\\
2	18.984375\\
2.1	19.51171875\\
2.2	19.951171875\\
2.3	20.390625\\
2.4	20.654296875\\
2.5	20.91796875\\
2.6	21.181640625\\
2.7	21.357421875\\
2.8	21.533203125\\
2.9	21.708984375\\
3	21.97265625\\
3.1	22.060546875\\
3.2	22.236328125\\
3.3	22.412109375\\
3.4	22.587890625\\
3.5	22.67578125\\
3.6	22.8515625\\
3.7	22.939453125\\
3.8	23.02734375\\
3.9	23.115234375\\
4	23.203125\\
4.1	23.203125\\
4.2	23.203125\\
4.3	23.203125\\
4.4	23.291015625\\
4.5	23.291015625\\
4.6	23.37890625\\
4.7	23.291015625\\
4.8	23.291015625\\
4.9	23.291015625\\
5	23.203125\\
5.1	23.203125\\
5.2	23.291015625\\
5.3	23.291015625\\
5.4	23.203125\\
5.5	23.115234375\\
5.6	23.02734375\\
5.7	22.8515625\\
5.8	22.763671875\\
5.9	22.67578125\\
6	22.587890625\\
6.1	22.32421875\\
6.2	21.26953125\\
6.3	19.51171875\\
6.4	17.40234375\\
6.5	15.29296875\\
6.6	13.623046875\\
6.7	12.392578125\\
6.8	11.6015625\\
6.9	10.8984375\\
7	10.37109375\\
7.1	9.755859375\\
7.2	9.052734375\\
7.3	8.4375\\
7.4	7.734375\\
7.5	7.20703125\\
7.6	6.767578125\\
7.7	6.328125\\
7.8	5.9765625\\
7.9	5.625\\
8	5.2734375\\
8.1	4.921875\\
8.2	4.74609375\\
8.3	4.482421875\\
8.4	4.306640625\\
8.5	4.21875\\
8.6	4.04296875\\
8.7	3.955078125\\
8.8	3.8671875\\
8.9	3.8671875\\
9	3.8671875\\
9.1	3.8671875\\
9.2	3.955078125\\
9.3	4.04296875\\
9.4	4.04296875\\
9.5	4.04296875\\
9.6	3.955078125\\
9.7	3.955078125\\
9.8	3.8671875\\
9.9	3.8671875\\
10	3.955078125\\
10.1	3.955078125\\
10.2	3.8671875\\
10.3	3.8671875\\
10.4	3.955078125\\
10.5	3.955078125\\
10.6	4.04296875\\
10.7	4.130859375\\
10.8	4.130859375\\
10.9	4.21875\\
11	4.130859375\\
11.1	4.04296875\\
11.2	3.955078125\\
11.3	3.8671875\\
11.4	3.779296875\\
11.5	3.779296875\\
11.6	3.779296875\\
11.7	3.69140625\\
11.8	3.69140625\\
11.9	3.603515625\\
12	3.603515625\\
12.1	3.603515625\\
12.2	3.603515625\\
12.3	3.603515625\\
12.4	3.69140625\\
12.5	3.69140625\\
12.6	3.69140625\\
12.7	3.779296875\\
12.8	3.779296875\\
12.9	3.779296875\\
13	3.779296875\\
13.1	3.779296875\\
13.2	3.779296875\\
13.3	3.779296875\\
13.4	3.69140625\\
13.5	3.69140625\\
13.6	3.69140625\\
13.7	3.69140625\\
13.8	3.69140625\\
13.9	3.69140625\\
14	3.779296875\\
14.1	3.779296875\\
14.2	3.69140625\\
14.3	3.69140625\\
14.4	3.779296875\\
14.5	3.779296875\\
14.6	3.779296875\\
14.7	3.8671875\\
14.8	3.8671875\\
14.9	3.8671875\\
15	3.955078125\\
15.1	3.955078125\\
15.2	3.955078125\\
15.3	4.04296875\\
15.4	4.04296875\\
15.5	4.130859375\\
15.6	4.130859375\\
15.7	4.130859375\\
15.8	4.130859375\\
15.9	4.130859375\\
16	4.04296875\\
16.1	4.04296875\\
16.2	4.04296875\\
16.3	4.04296875\\
16.4	4.04296875\\
16.5	4.04296875\\
16.6	4.130859375\\
16.7	4.130859375\\
16.8	4.130859375\\
16.9	4.130859375\\
17	4.130859375\\
17.1	4.130859375\\
17.2	4.130859375\\
17.3	4.130859375\\
17.4	4.04296875\\
17.5	3.955078125\\
17.6	3.8671875\\
17.7	3.8671875\\
17.8	3.779296875\\
17.9	3.779296875\\
18	3.779296875\\
18.1	3.69140625\\
18.2	3.69140625\\
18.3	3.69140625\\
18.4	3.69140625\\
18.5	3.779296875\\
18.6	3.779296875\\
18.7	3.8671875\\
};
\addlegendentry{p}

\end{axis}

\begin{axis}[%
width=0.951\figurewidth,
height=0.419\figureheight,
at={(0\figurewidth,0\figureheight)},
scale only axis,
xmin=0,
xmax=14,
xlabel={Time [s]},
ymin=-30,
ymax=5,
ylabel={e [deg]},
axis background/.style={fill=white},
legend style={legend cell align=left, align=left, draw=black}
]
\addplot [color=blue]
  table[row sep=crcr]{%
0	0\\
};
%\addlegendentry{$\text{e}_{\text{ref}}$}

\addplot [color=black!50!green]
  table[row sep=crcr]{%
0.1	0\\
};
%\addlegendentry{e}

\addplot [color=red, forget plot]
  table[row sep=crcr]{%
0.2	0\\
};
\addplot [color=mycolor1, forget plot]
  table[row sep=crcr]{%
0.3	0\\
};
\addplot [color=mycolor2, forget plot]
  table[row sep=crcr]{%
0.4	0\\
};
\addplot [color=mycolor3, forget plot]
  table[row sep=crcr]{%
0.5	0\\
};
\addplot [color=darkgray, forget plot]
  table[row sep=crcr]{%
0.6	0\\
};
\addplot [color=blue, forget plot]
  table[row sep=crcr]{%
0.7	0\\
};
\addplot [color=black!50!green, forget plot]
  table[row sep=crcr]{%
0.8	0\\
};
\addplot [color=red, forget plot]
  table[row sep=crcr]{%
0.9	0\\
};
\addplot [color=mycolor1, forget plot]
  table[row sep=crcr]{%
1	0\\
};
\addplot [color=mycolor2, forget plot]
  table[row sep=crcr]{%
1.1	0\\
};
\addplot [color=mycolor3, forget plot]
  table[row sep=crcr]{%
1.2	0\\
};
\addplot [color=darkgray, forget plot]
  table[row sep=crcr]{%
1.3	0\\
};
\addplot [color=blue, forget plot]
  table[row sep=crcr]{%
1.4	0\\
};
\addplot [color=black!50!green, forget plot]
  table[row sep=crcr]{%
1.5	0\\
};
\addplot [color=red, forget plot]
  table[row sep=crcr]{%
1.6	0\\
};
\addplot [color=mycolor1, forget plot]
  table[row sep=crcr]{%
1.7	0\\
};
\addplot [color=mycolor2, forget plot]
  table[row sep=crcr]{%
1.8	0\\
};
\addplot [color=mycolor3, forget plot]
  table[row sep=crcr]{%
1.9	0\\
};
\addplot [color=darkgray, forget plot]
  table[row sep=crcr]{%
2	0\\
};
\addplot [color=blue, forget plot]
  table[row sep=crcr]{%
2.1	0\\
};
\addplot [color=black!50!green, forget plot]
  table[row sep=crcr]{%
2.2	0\\
};
\addplot [color=red, forget plot]
  table[row sep=crcr]{%
2.3	0\\
};
\addplot [color=mycolor1, forget plot]
  table[row sep=crcr]{%
2.4	0\\
};
\addplot [color=mycolor2, forget plot]
  table[row sep=crcr]{%
2.5	0\\
};
\addplot [color=mycolor3, forget plot]
  table[row sep=crcr]{%
2.6	0\\
};
\addplot [color=darkgray, forget plot]
  table[row sep=crcr]{%
2.7	0\\
};
\addplot [color=blue, forget plot]
  table[row sep=crcr]{%
2.8	0\\
};
\addplot [color=black!50!green, forget plot]
  table[row sep=crcr]{%
2.9	0\\
};
\addplot [color=red, forget plot]
  table[row sep=crcr]{%
3	0\\
};
\addplot [color=mycolor1, forget plot]
  table[row sep=crcr]{%
3.1	0\\
};
\addplot [color=mycolor2, forget plot]
  table[row sep=crcr]{%
3.2	0\\
};
\addplot [color=mycolor3, forget plot]
  table[row sep=crcr]{%
3.3	0\\
};
\addplot [color=darkgray, forget plot]
  table[row sep=crcr]{%
3.4	0\\
};
\addplot [color=blue, forget plot]
  table[row sep=crcr]{%
3.5	0\\
};
\addplot [color=black!50!green, forget plot]
  table[row sep=crcr]{%
3.6	0\\
};
\addplot [color=red, forget plot]
  table[row sep=crcr]{%
3.7	0\\
};
\addplot [color=mycolor1, forget plot]
  table[row sep=crcr]{%
3.8	0\\
};
\addplot [color=mycolor2, forget plot]
  table[row sep=crcr]{%
3.9	0\\
};
\addplot [color=mycolor3, forget plot]
  table[row sep=crcr]{%
4	0\\
};
\addplot [color=darkgray, forget plot]
  table[row sep=crcr]{%
4.1	0\\
};
\addplot [color=blue, forget plot]
  table[row sep=crcr]{%
4.2	0\\
};
\addplot [color=black!50!green, forget plot]
  table[row sep=crcr]{%
4.3	0\\
};
\addplot [color=red, forget plot]
  table[row sep=crcr]{%
4.4	0\\
};
\addplot [color=mycolor1, forget plot]
  table[row sep=crcr]{%
4.5	0\\
};
\addplot [color=mycolor2, forget plot]
  table[row sep=crcr]{%
4.6	0\\
};
\addplot [color=mycolor3, forget plot]
  table[row sep=crcr]{%
4.7	0\\
};
\addplot [color=darkgray, forget plot]
  table[row sep=crcr]{%
4.8	0\\
};
\addplot [color=blue, forget plot]
  table[row sep=crcr]{%
4.9	0\\
};
\addplot [color=black!50!green, forget plot]
  table[row sep=crcr]{%
5	0\\
};
\addplot [color=red, forget plot]
  table[row sep=crcr]{%
5.1	0\\
};
\addplot [color=mycolor1, forget plot]
  table[row sep=crcr]{%
5.2	0\\
};
\addplot [color=mycolor2, forget plot]
  table[row sep=crcr]{%
5.3	0\\
};
\addplot [color=mycolor3, forget plot]
  table[row sep=crcr]{%
5.4	0\\
};
\addplot [color=darkgray, forget plot]
  table[row sep=crcr]{%
5.5	0\\
};
\addplot [color=blue, forget plot]
  table[row sep=crcr]{%
5.6	0\\
};
\addplot [color=black!50!green, forget plot]
  table[row sep=crcr]{%
5.7	0\\
};
\addplot [color=red, forget plot]
  table[row sep=crcr]{%
5.8	0\\
};
\addplot [color=mycolor1, forget plot]
  table[row sep=crcr]{%
5.9	0\\
};
\addplot [color=mycolor2, forget plot]
  table[row sep=crcr]{%
6	0\\
};
\addplot [color=mycolor3, forget plot]
  table[row sep=crcr]{%
6.1	0\\
};
\addplot [color=darkgray, forget plot]
  table[row sep=crcr]{%
6.2	0\\
};
\addplot [color=blue, forget plot]
  table[row sep=crcr]{%
6.3	0\\
};
\addplot [color=black!50!green, forget plot]
  table[row sep=crcr]{%
6.4	0\\
};
\addplot [color=red, forget plot]
  table[row sep=crcr]{%
6.5	0\\
};
\addplot [color=mycolor1, forget plot]
  table[row sep=crcr]{%
6.6	0\\
};
\addplot [color=mycolor2, forget plot]
  table[row sep=crcr]{%
6.7	0\\
};
\addplot [color=mycolor3, forget plot]
  table[row sep=crcr]{%
6.8	0\\
};
\addplot [color=darkgray, forget plot]
  table[row sep=crcr]{%
6.9	0\\
};
\addplot [color=blue, forget plot]
  table[row sep=crcr]{%
7	0\\
};
\addplot [color=black!50!green, forget plot]
  table[row sep=crcr]{%
7.1	0\\
};
\addplot [color=red, forget plot]
  table[row sep=crcr]{%
7.2	0\\
};
\addplot [color=mycolor1, forget plot]
  table[row sep=crcr]{%
7.3	0\\
};
\addplot [color=mycolor2, forget plot]
  table[row sep=crcr]{%
7.4	0\\
};
\addplot [color=mycolor3, forget plot]
  table[row sep=crcr]{%
7.5	0\\
};
\addplot [color=darkgray, forget plot]
  table[row sep=crcr]{%
7.6	0\\
};
\addplot [color=blue, forget plot]
  table[row sep=crcr]{%
7.7	0\\
};
\addplot [color=black!50!green, forget plot]
  table[row sep=crcr]{%
7.8	0\\
};
\addplot [color=red, forget plot]
  table[row sep=crcr]{%
7.9	0\\
};
\addplot [color=mycolor1, forget plot]
  table[row sep=crcr]{%
8	0\\
};
\addplot [color=mycolor2, forget plot]
  table[row sep=crcr]{%
8.1	0\\
};
\addplot [color=mycolor3, forget plot]
  table[row sep=crcr]{%
8.2	0\\
};
\addplot [color=darkgray, forget plot]
  table[row sep=crcr]{%
8.3	0\\
};
\addplot [color=blue, forget plot]
  table[row sep=crcr]{%
8.4	0\\
};
\addplot [color=black!50!green, forget plot]
  table[row sep=crcr]{%
8.5	0\\
};
\addplot [color=red, forget plot]
  table[row sep=crcr]{%
8.6	0\\
};
\addplot [color=mycolor1, forget plot]
  table[row sep=crcr]{%
8.7	0\\
};
\addplot [color=mycolor2, forget plot]
  table[row sep=crcr]{%
8.8	0\\
};
\addplot [color=mycolor3, forget plot]
  table[row sep=crcr]{%
8.9	0\\
};
\addplot [color=darkgray, forget plot]
  table[row sep=crcr]{%
9	0\\
};
\addplot [color=blue, forget plot]
  table[row sep=crcr]{%
9.1	0\\
};
\addplot [color=black!50!green, forget plot]
  table[row sep=crcr]{%
9.2	0\\
};
\addplot [color=red, forget plot]
  table[row sep=crcr]{%
9.3	0\\
};
\addplot [color=mycolor1, forget plot]
  table[row sep=crcr]{%
9.4	0\\
};
\addplot [color=mycolor2, forget plot]
  table[row sep=crcr]{%
9.5	0\\
};
\addplot [color=mycolor3, forget plot]
  table[row sep=crcr]{%
9.6	0\\
};
\addplot [color=darkgray, forget plot]
  table[row sep=crcr]{%
9.7	0\\
};
\addplot [color=blue, forget plot]
  table[row sep=crcr]{%
9.8	0\\
};
\addplot [color=black!50!green, forget plot]
  table[row sep=crcr]{%
9.9	0\\
};
\addplot [color=red, forget plot]
  table[row sep=crcr]{%
10	0\\
};
\addplot [color=mycolor1, forget plot]
  table[row sep=crcr]{%
10.1	0\\
};
\addplot [color=mycolor2, forget plot]
  table[row sep=crcr]{%
10.2	0\\
};
\addplot [color=mycolor3, forget plot]
  table[row sep=crcr]{%
10.3	0\\
};
\addplot [color=darkgray, forget plot]
  table[row sep=crcr]{%
10.4	0\\
};
\addplot [color=blue, forget plot]
  table[row sep=crcr]{%
10.5	0\\
};
\addplot [color=black!50!green, forget plot]
  table[row sep=crcr]{%
10.6	0\\
};
\addplot [color=red, forget plot]
  table[row sep=crcr]{%
10.7	0\\
};
\addplot [color=mycolor1, forget plot]
  table[row sep=crcr]{%
10.8	0\\
};
\addplot [color=mycolor2, forget plot]
  table[row sep=crcr]{%
10.9	0\\
};
\addplot [color=mycolor3, forget plot]
  table[row sep=crcr]{%
11	0\\
};
\addplot [color=darkgray, forget plot]
  table[row sep=crcr]{%
11.1	0\\
};
\addplot [color=blue, forget plot]
  table[row sep=crcr]{%
11.2	0\\
};
\addplot [color=black!50!green, forget plot]
  table[row sep=crcr]{%
11.3	0\\
};
\addplot [color=red, forget plot]
  table[row sep=crcr]{%
11.4	0\\
};
\addplot [color=mycolor1, forget plot]
  table[row sep=crcr]{%
11.5	0\\
};
\addplot [color=mycolor2, forget plot]
  table[row sep=crcr]{%
11.6	0\\
};
\addplot [color=mycolor3, forget plot]
  table[row sep=crcr]{%
11.7	0\\
};
\addplot [color=darkgray, forget plot]
  table[row sep=crcr]{%
11.8	0\\
};
\addplot [color=blue, forget plot]
  table[row sep=crcr]{%
11.9	0\\
};
\addplot [color=black!50!green, forget plot]
  table[row sep=crcr]{%
12	0\\
};
\addplot [color=red, forget plot]
  table[row sep=crcr]{%
12.1	0\\
};
\addplot [color=mycolor1, forget plot]
  table[row sep=crcr]{%
12.2	0\\
};
\addplot [color=mycolor2, forget plot]
  table[row sep=crcr]{%
12.3	0\\
};
\addplot [color=mycolor3, forget plot]
  table[row sep=crcr]{%
12.4	0\\
};
\addplot [color=darkgray, forget plot]
  table[row sep=crcr]{%
12.5	0\\
};
\addplot [color=blue, forget plot]
  table[row sep=crcr]{%
12.6	0\\
};
\addplot [color=black!50!green, forget plot]
  table[row sep=crcr]{%
12.7	0\\
};
\addplot [color=red, forget plot]
  table[row sep=crcr]{%
12.8	0\\
};
\addplot [color=mycolor1, forget plot]
  table[row sep=crcr]{%
12.9	0\\
};
\addplot [color=mycolor2, forget plot]
  table[row sep=crcr]{%
13	0\\
};
\addplot [color=mycolor3, forget plot]
  table[row sep=crcr]{%
13.1	0\\
};
\addplot [color=darkgray, forget plot]
  table[row sep=crcr]{%
13.2	0\\
};
\addplot [color=blue, forget plot]
  table[row sep=crcr]{%
13.3	0\\
};
\addplot [color=black!50!green, forget plot]
  table[row sep=crcr]{%
13.4	0\\
};
\addplot [color=red, forget plot]
  table[row sep=crcr]{%
13.5	0\\
};
\addplot [color=mycolor1, forget plot]
  table[row sep=crcr]{%
13.6	0\\
};
\addplot [color=mycolor2, forget plot]
  table[row sep=crcr]{%
13.7	0\\
};
\addplot [color=mycolor3, forget plot]
  table[row sep=crcr]{%
13.8	0\\
};
\addplot [color=darkgray, forget plot]
  table[row sep=crcr]{%
13.9	0\\
};
\addplot [color=blue, forget plot]
  table[row sep=crcr]{%
14	0\\
};
\addplot [color=black!50!green, forget plot]
  table[row sep=crcr]{%
14.1	0\\
};
\addplot [color=red, forget plot]
  table[row sep=crcr]{%
14.2	0\\
};
\addplot [color=mycolor1, forget plot]
  table[row sep=crcr]{%
14.3	0\\
};
\addplot [color=mycolor2, forget plot]
  table[row sep=crcr]{%
14.4	0\\
};
\addplot [color=mycolor3, forget plot]
  table[row sep=crcr]{%
14.5	0\\
};
\addplot [color=darkgray, forget plot]
  table[row sep=crcr]{%
14.6	0\\
};
\addplot [color=blue, forget plot]
  table[row sep=crcr]{%
14.7	0\\
};
\addplot [color=black!50!green, forget plot]
  table[row sep=crcr]{%
14.8	0\\
};
\addplot [color=red, forget plot]
  table[row sep=crcr]{%
14.9	0\\
};
\addplot [color=mycolor1, forget plot]
  table[row sep=crcr]{%
15	0\\
};
\addplot [color=mycolor2, forget plot]
  table[row sep=crcr]{%
15.1	0\\
};
\addplot [color=mycolor3, forget plot]
  table[row sep=crcr]{%
15.2	0\\
};
\addplot [color=darkgray, forget plot]
  table[row sep=crcr]{%
15.3	0\\
};
\addplot [color=blue, forget plot]
  table[row sep=crcr]{%
15.4	0\\
};
\addplot [color=black!50!green, forget plot]
  table[row sep=crcr]{%
15.5	0\\
};
\addplot [color=red, forget plot]
  table[row sep=crcr]{%
15.6	0\\
};
\addplot [color=mycolor1, forget plot]
  table[row sep=crcr]{%
15.7	0\\
};
\addplot [color=mycolor2, forget plot]
  table[row sep=crcr]{%
15.8	0\\
};
\addplot [color=mycolor3, forget plot]
  table[row sep=crcr]{%
15.9	0\\
};
\addplot [color=darkgray, forget plot]
  table[row sep=crcr]{%
16	0\\
};
\addplot [color=blue, forget plot]
  table[row sep=crcr]{%
16.1	0\\
};
\addplot [color=black!50!green, forget plot]
  table[row sep=crcr]{%
16.2	0\\
};
\addplot [color=red, forget plot]
  table[row sep=crcr]{%
16.3	0\\
};
\addplot [color=mycolor1, forget plot]
  table[row sep=crcr]{%
16.4	0\\
};
\addplot [color=mycolor2, forget plot]
  table[row sep=crcr]{%
16.5	0\\
};
\addplot [color=mycolor3, forget plot]
  table[row sep=crcr]{%
16.6	0\\
};
\addplot [color=darkgray, forget plot]
  table[row sep=crcr]{%
16.7	0\\
};
\addplot [color=blue, forget plot]
  table[row sep=crcr]{%
16.8	0\\
};
\addplot [color=black!50!green, forget plot]
  table[row sep=crcr]{%
16.9	0\\
};
\addplot [color=red, forget plot]
  table[row sep=crcr]{%
17	0\\
};
\addplot [color=mycolor1, forget plot]
  table[row sep=crcr]{%
17.1	0\\
};
\addplot [color=mycolor2, forget plot]
  table[row sep=crcr]{%
17.2	0\\
};
\addplot [color=mycolor3, forget plot]
  table[row sep=crcr]{%
17.3	0\\
};
\addplot [color=darkgray, forget plot]
  table[row sep=crcr]{%
17.4	0\\
};
\addplot [color=blue, forget plot]
  table[row sep=crcr]{%
17.5	0\\
};
\addplot [color=black!50!green, forget plot]
  table[row sep=crcr]{%
17.6	0\\
};
\addplot [color=red, forget plot]
  table[row sep=crcr]{%
17.7	0\\
};
\addplot [color=mycolor1, forget plot]
  table[row sep=crcr]{%
17.8	0\\
};
\addplot [color=mycolor2, forget plot]
  table[row sep=crcr]{%
17.9	0\\
};
\addplot [color=mycolor3, forget plot]
  table[row sep=crcr]{%
18	0\\
};
\addplot [color=darkgray, forget plot]
  table[row sep=crcr]{%
18.1	0\\
};
\addplot [color=blue, forget plot]
  table[row sep=crcr]{%
18.2	0\\
};
\addplot [color=black!50!green, forget plot]
  table[row sep=crcr]{%
18.3	0\\
};
\addplot [color=red, forget plot]
  table[row sep=crcr]{%
18.4	0\\
};
\addplot [color=mycolor1, forget plot]
  table[row sep=crcr]{%
18.5	0\\
};
\addplot [color=mycolor2, forget plot]
  table[row sep=crcr]{%
18.6	0\\
};
\addplot [color=mycolor3, forget plot]
  table[row sep=crcr]{%
18.7	0\\
};
\addplot [color=darkgray, forget plot]
  table[row sep=crcr]{%
0	-30\\
0.1	-29.912109375\\
0.2	-29.82421875\\
0.3	-29.560546875\\
0.4	-28.857421875\\
0.5	-27.71484375\\
0.6	-26.220703125\\
0.7	-24.462890625\\
0.8	-22.529296875\\
0.9	-20.5078125\\
1	-18.57421875\\
1.1	-16.640625\\
1.2	-14.794921875\\
1.3	-13.037109375\\
1.4	-11.455078125\\
1.5	-9.9609375\\
1.6	-8.73046875\\
1.7	-7.587890625\\
1.8	-6.62109375\\
1.9	-5.830078125\\
2	-5.126953125\\
2.1	-4.599609375\\
2.2	-4.16015625\\
2.3	-3.80859375\\
2.4	-3.544921875\\
2.5	-3.28125\\
2.6	-3.193359375\\
2.7	-3.017578125\\
2.8	-2.841796875\\
2.9	-2.75390625\\
3	-2.666015625\\
3.1	-2.490234375\\
3.2	-2.40234375\\
3.3	-2.314453125\\
3.4	-2.2265625\\
3.5	-2.138671875\\
3.6	-2.05078125\\
3.7	-1.962890625\\
3.8	-1.875\\
3.9	-1.787109375\\
4	-1.69921875\\
4.1	-1.611328125\\
4.2	-1.5234375\\
4.3	-1.435546875\\
4.4	-1.34765625\\
4.5	-1.259765625\\
4.6	-1.171875\\
4.7	-1.083984375\\
4.8	-1.083984375\\
4.9	-0.996093750000004\\
5	-0.908203125\\
5.1	-0.820312499999996\\
5.2	-0.732421875\\
5.3	-0.644531249999996\\
5.4	-0.556640625\\
5.5	-0.556640625\\
5.6	-0.46875\\
5.7	-0.380859375000004\\
5.8	-0.380859375000004\\
5.9	-0.29296875\\
6	-0.29296875\\
6.1	-0.205078125000004\\
6.2	-0.205078125000004\\
6.3	-0.1171875\\
6.4	-0.1171875\\
6.5	-0.0292968750000036\\
6.6	-0.0292968750000036\\
6.7	0.05859375\\
6.8	0.146484375000004\\
6.9	0.234375\\
7	0.322265625000004\\
7.1	0.41015625\\
7.2	0.498046875\\
7.3	0.585937499999996\\
7.4	0.673828125\\
7.5	0.673828125\\
7.6	0.761718749999996\\
7.7	0.849609375\\
7.8	0.937499999999996\\
7.9	0.937499999999996\\
8	1.025390625\\
8.1	1.11328125\\
8.2	1.11328125\\
8.3	1.11328125\\
8.4	1.201171875\\
8.5	1.201171875\\
8.6	1.201171875\\
8.7	1.201171875\\
8.8	1.2890625\\
8.9	1.2890625\\
9	1.2890625\\
9.1	1.2890625\\
9.2	1.2890625\\
9.3	1.2890625\\
9.4	1.201171875\\
9.5	1.201171875\\
9.6	1.201171875\\
9.7	1.201171875\\
9.8	1.201171875\\
9.9	1.201171875\\
10	1.11328125\\
10.1	1.11328125\\
10.2	1.025390625\\
10.3	1.025390625\\
10.4	0.937499999999996\\
10.5	0.849609375\\
10.6	0.761718749999996\\
10.7	0.673828125\\
10.8	0.673828125\\
10.9	0.585937499999996\\
11	0.498046875\\
11.1	0.498046875\\
11.2	0.41015625\\
11.3	0.41015625\\
11.4	0.322265625000004\\
11.5	0.322265625000004\\
11.6	0.322265625000004\\
11.7	0.322265625000004\\
11.8	0.234375\\
11.9	0.234375\\
12	0.234375\\
12.1	0.234375\\
12.2	0.146484375000004\\
12.3	0.146484375000004\\
12.4	0.146484375000004\\
12.5	0.05859375\\
12.6	0.05859375\\
12.7	-0.0292968750000036\\
12.8	-0.0292968750000036\\
12.9	-0.0292968750000036\\
13	-0.0292968750000036\\
13.1	-0.0292968750000036\\
13.2	-0.0292968750000036\\
13.3	0.05859375\\
13.4	0.05859375\\
13.5	-0.0292968750000036\\
13.6	-0.0292968750000036\\
13.7	-0.0292968750000036\\
13.8	-0.0292968750000036\\
13.9	-0.0292968750000036\\
14	-0.1171875\\
14.1	-0.1171875\\
14.2	-0.1171875\\
14.3	-0.1171875\\
14.4	-0.205078125000004\\
14.5	-0.205078125000004\\
14.6	-0.205078125000004\\
14.7	-0.205078125000004\\
14.8	-0.205078125000004\\
14.9	-0.29296875\\
15	-0.29296875\\
15.1	-0.29296875\\
15.2	-0.29296875\\
15.3	-0.29296875\\
15.4	-0.29296875\\
15.5	-0.380859375000004\\
15.6	-0.380859375000004\\
15.7	-0.380859375000004\\
15.8	-0.380859375000004\\
15.9	-0.380859375000004\\
16	-0.380859375000004\\
16.1	-0.380859375000004\\
16.2	-0.380859375000004\\
16.3	-0.380859375000004\\
16.4	-0.380859375000004\\
16.5	-0.380859375000004\\
16.6	-0.380859375000004\\
16.7	-0.380859375000004\\
16.8	-0.380859375000004\\
16.9	-0.380859375000004\\
17	-0.380859375000004\\
17.1	-0.380859375000004\\
17.2	-0.380859375000004\\
17.3	-0.380859375000004\\
17.4	-0.380859375000004\\
17.5	-0.380859375000004\\
17.6	-0.380859375000004\\
17.7	-0.380859375000004\\
17.8	-0.380859375000004\\
17.9	-0.46875\\
18	-0.46875\\
18.1	-0.46875\\
18.2	-0.46875\\
18.3	-0.46875\\
18.4	-0.46875\\
18.5	-0.556640625\\
18.6	-0.556640625\\
18.7	-0.556640625\\
};
\end{axis}
\end{tikzpicture}%
        \caption{First flight testing the pitch PD-controller and the elevation PID-controller, the elevation reference is zero degrees.} 
\label{fig:repetition} 
\end{figure}

As we can see in \cref{fig:repetition} the elevation reaches zero degrees as it should, but the pitch, which is controlled by a PD-controller has a few degrees standard deviation. This is intended to prevent drift in the travel caused by the fact that both rotors rotate in the same direction, thus creating a torque causing the helicopter to travel. Thus the standard pitch deviation is in fact a feed foreword to eliminate standard deviation in travel speed.\\
\\
Both controllers gave satisfactory performance, and provided a good base for more advanced control engineering.