\section{Optimal Control in Two Dimensions}\label{sec:two_dimensions}
We now want to extend our system to be able to fly over an obstacle as the helicopter moves. To facilitate this, we need control over the elevation as well as the travel. We therefore expand our input vector to include the manipulated variable $e_c$, which is the reference for elevation. This also gives us a new state vector, namely $x = [\lambda\, r\, p\, \dot{p}\, e\, \dot{e}]^\top$. Coupled with our new input, $u = [p_c\, e_c]^\top$, we obtain the new state space model for our helicopter seen in equation \ref{eq:two_dim_ss}. The subscript $c$ in equations \ref{eq:two_dim_A_c} and \ref{eq:two_dim_B_c} denote that these are time continuous matrices.
\begin{subequations}\label{eq:two_dim_ss}
\begin{align}
    \label{eq:two_dim_A_c}
    A_c &= \begin{bmatrix}
        0 & 1 & 0 & 0 & 0 & 0\\
        0 & 0 & -K_2 & 0 & 0 & 0\\
        0 & 0 & 0 & 1 & 0 & 0\\
        0 & 0 & -K_1K_{pp} & -K_1K_{pd} & 0 & 0\\
        0 & 0 & 0 & 0 & 0 & 1\\
        0 & 0 & 0 & 0 & -K_3K_{ep} & -K_3K_{ed}
    \end{bmatrix}\\
    \label{eq:two_dim_B_c}
    B_c &= \begin{bmatrix}
        0 & 0\\
        0 & 0\\
        0 & 0\\
        K_1K_{pp} & 0\\
        0 & 0\\
        0 & K_3K_{ep}
    \end{bmatrix}
\end{align}
\end{subequations}
To discretize our helicopter model using forward Euler, we can use the MATLAB code listed in listing \ref{lst:two_dim_con_to_dis}.
\begin{lstlisting}[caption=Discretization of the continuous model in equation \ref{eq:two_dim_ss}.,label=lst:two_dim_con_to_dis]
%% Time step size 0.25 seconds
delta_t = 0.25;
%% Define continuous system
A_c = [0, 1, 0, 0, 0, 0;
       0, 0, -K_2, 0, 0, 0;
       0, 0, 0, 1, 0, 0;
       0, 0, -K_1*K_pp, -K_1*K_pd, 0, 0;
       0, 0, 0, 0, 0, 1;
       0, 0, 0, 0, -K_3*K_ep, -K_3*K_ed];
   
B_c = [0, 0;
       0, 0;
       0, 0;
       K_1*K_pp, 0;
       0, 0;
       0, K_3*K_ep];
%% Discretize the continuous system
A = eye(6) + delta_t * A_c;
B = delta_t * B_c;
\end{lstlisting}
To implement the obstacle, we want our helicopter to abide by the constraint given in equation \ref{eq:two_dim_obstacle}. This defines a small "mountain" for the helicopter to fly over.
\begin{equation}\label{eq:two_dim_obstacle}
    e_k \geq \alpha e^{-\beta (\lambda_k - \lambda_t)^2}\ \forall k\in [1,N]
\end{equation}
In Quadratic Programming, a quadratic objective function with linear constraints is solved. Since equation \ref{eq:two_dim_obstacle} is clearly not a linear constraint, a QP solver is unusable in this case. However, MATLAB includes the function \lstinline!fmincon! which can minimize an objective function with an arbitrary constraint function.\\
\\
To implement the constraint from equation \ref{eq:two_dim_obstacle} in MATLAB, we defined a seperate function, called \lstinline!nonlcon!. This function can be seen in listing \ref{lst:nonlcon}.
\begin{lstlisting}[caption=MATLAB implementation of non-linear "mountain" constraint.,label=lst:nonlcon]
function [c, c_eq] = nonlcon(z)
%% alpha, beta, and lambda are given
alpha = 0.2;
beta = 20;
lambda_t = 2 * pi / 3;

%% extract time horizon size from input
U = size(z, 1)  / 8;

%% extract travel and elevation from input
lambda = z(1:6:6*U);
e = z(5:6:6*U);

c = zeros(U, 1);

for n = 1:U
    c(n) = alpha * exp(-beta * (lambda(n) - lambda_t).^2) - e(n);
end

c_eq = [];

end
\end{lstlisting}
Our new objective function, under the mountain constraint, is seen in equation \ref{eq:two_dim_objective_function}. In addition to this, the "discrete mountain constraints" have to be passed to \lstinline!fmincon! - that is, \lstinline!fmincon! needs to take into consideration the elevation constraint $c(x_k)\,\forall k\in [1,N]$, where $N$ is the time horizon.
\begin{equation}\label{eq:two_dim_objective_function}
    \phi = \sum_{i = 1}^N \bigg(\lambda_i - \lambda_f\bigg)^2 + q_1p_{ci}\,^2+q_2e_{ci}\,^2
\end{equation}
The setup of \lstinline!fmincon! to solve our new objective function with the imposed mountain constraint can be seen in listing \ref{lst:fmincon_setup}. Here, \lstinline!@nonlcon! is provided as a separate file, as seen in listing \ref{lst:nonlcon}.
\begin{lstlisting}[caption=Setup of \lstinline!fmincon! for objective function \ref{eq:two_dim_objective_function}.,label=lst:fmincon_setup]
%% Objective function
objective = @(z) z'*Q*z;
options = optimset('MaxFunEvals',60000,'Algorithm',  'active-set');
z = fmincon(objective, z0,[],[],Aeq,beq,vlb,vub,@nonlcon,options);
\end{lstlisting}
\lstinline!Aeq! and \lstinline!Beq! in listing \ref{lst:fmincon_setup} are essentially equal to \lstinline!A! and \lstinline!B! from listing \ref{lst:two_dim_con_to_dis}, but repeated after each other, corresponding to the $40$ time steps \lstinline!fmincon! operates over. \lstinline!vlb! and \lstinline!vub! define the physical constraints for the pitch and elevation; that is, $-35^\circ \leq e \leq 35^\circ$ and $-60^\circ \leq p \leq 60^\circ$. Furthermore, \lstinline!z0! is simply a zero vector used by \lstinline!nonlcon!. Finally, the \lstinline!Q! matrix defines the importance of each state and drive penalties, and was computed as in listing \ref{lst:define_Q_and_q}.
\begin{lstlisting}[caption=Definition of Q matrix and drive penalties.,label=lst:define_Q_and_q]
%% Generate the matrix Q
Q1 = zeros(mx,mx);
Q1(1,1) = 0.5;       % Weight on travel
Q1(2,2) = 1;         % Weight on travel dot
Q1(3,3) = 0;         % Weight on pitch
Q1(4,4) = 1;         % Weight on pitch dot
Q1(5,5) = 1;         % Weight on elevation
Q1(6,6) = 1;         % Weight on elevation dot

q1 = 0.5;            % Weight on input pitch
q2 = 0.5;            % Weight on input elevation

Q = 2*genq2(Q1,[q1, 0; 0, q2],N,M,mu); % Generate Q
\end{lstlisting}
The optimal control input sequence $u^*$ and optimal trajectory $x^*$ were extracted from the result returned by \lstinline!fmincon! in listing \ref{lst:fmincon_setup}. In \cref{fig:mountain} we can see that the optimal trajectory barley avoids the mountain, as expected. 
\begin{figure} 
        \centering
        \setlength{\figureheight}{6cm}
        \setlength{\figurewidth}{10cm}
        % This file was created by matlab2tikz.
%
%The latest updates can be retrieved from
%  http://www.mathworks.com/matlabcentral/fileexchange/22022-matlab2tikz-matlab2tikz
%where you can also make suggestions and rate matlab2tikz.
%
\begin{tikzpicture}

\begin{axis}[%
width=0.951\figurewidth,
height=\figureheight,
at={(0\figurewidth,0\figureheight)},
scale only axis,
every outer x axis line/.append style={black},
every x tick label/.append style={font=\color{black}},
every x tick/.append style={black},
xmin=-20,
xmax=180,
xlabel={$\lambda$/degrees},
every outer y axis line/.append style={black},
every y tick label/.append style={font=\color{black}},
every y tick/.append style={black},
ymin=0,
ymax=12,
ylabel={$e$/degrees},
axis background/.style={fill=white},
axis x line*=bottom,
axis y line*=left,
legend style={at={(0.726,0.821)}, anchor=south west, legend cell align=left, align=left, draw=black}
]
\addplot [color=red]
  table[row sep=crcr]{%
180	3.41986359203811e-09\\
180	3.41986359203811e-09\\
180	3.41986359203811e-09\\
180	3.41986359203811e-09\\
180	3.41986359203811e-09\\
180	3.41986359203811e-09\\
180	3.41986359203811e-09\\
180	3.41986359203811e-09\\
180	3.41986359203811e-09\\
180	3.41986359203811e-09\\
180	3.41986359203811e-09\\
180	3.41986359203811e-09\\
180	3.41986359203811e-09\\
180	3.41986359203811e-09\\
180	3.41986359203811e-09\\
180	3.41986359203811e-09\\
180	3.41986359203811e-09\\
180	3.41986359203811e-09\\
180	3.41986359203811e-09\\
180	3.41986359203811e-09\\
180	3.41986359203811e-09\\
180	3.41986359203811e-09\\
180	3.41986359203811e-09\\
180	3.41986359203811e-09\\
179.677667217548	4.32589611321541e-09\\
178.678435591947	8.89182109990403e-09\\
176.709788123122	3.54832952247894e-08\\
173.557212344351	2.9498222370491e-07\\
169.073230934763	4.8686461499792e-06\\
163.160508862583	0.000135002478931176\\
155.852118925089	0.00455213100659387\\
147.297486702545	0.122341215121836\\
137.723280312707	1.690616198467\\
127.397369978029	8.21046165636209\\
116.599724330969	10.6797570407617\\
105.59927425757	3.23928409662421\\
94.6361890518527	0.227510651805454\\
83.9160380767513	0.00411232285875604\\
73.6079516077997	2.31550389769611e-05\\
63.8451250889702	5.19679079599913e-08\\
54.7267828593082	6.11172405759611e-11\\
46.3210136687073	4.96344639208469e-14\\
38.6681177920476	3.6060469781914e-17\\
31.7842603637649	2.94389588605723e-20\\
25.6652504754591	3.26142383090712e-23\\
20.2903212181664	5.67319024237427e-26\\
15.6257560759077	1.71812409609218e-28\\
11.6282665208202	9.6576632629194e-31\\
8.24803650559684	1.03791188815312e-32\\
5.43138099753073	2.13551817864177e-34\\
3.12301363188964	8.23976995043114e-36\\
1.26791468210254	5.74727140572494e-37\\
-0.187164272560405	6.91247313670747e-38\\
-1.2925352025654	1.35951712077853e-38\\
-2.09469718340566	4.13823361951519e-39\\
-2.6357146398433	1.84710812697897e-39\\
-2.95281914610442	1.14931489299953e-39\\
-3.07823088358906	9.52357779839886e-40\\
-3.03918688551579	1.00977684983918e-39\\
-2.85814365922469	1.32438580976805e-39\\
-2.55312893077553	2.08962741522026e-39\\
0	9.09027979104266e-38\\
0	9.09027979104266e-38\\
0	9.09027979104266e-38\\
0	9.09027979104266e-38\\
0	9.09027979104266e-38\\
0	9.09027979104266e-38\\
0	9.09027979104266e-38\\
0	9.09027979104266e-38\\
0	9.09027979104266e-38\\
0	9.09027979104266e-38\\
0	9.09027979104266e-38\\
0	9.09027979104266e-38\\
0	9.09027979104266e-38\\
0	9.09027979104266e-38\\
0	9.09027979104266e-38\\
0	9.09027979104266e-38\\
0	9.09027979104266e-38\\
0	9.09027979104266e-38\\
0	9.09027979104266e-38\\
0	9.09027979104266e-38\\
};
\addlegendentry{Mountain}

\addplot [color=blue]
  table[row sep=crcr]{%
180	0\\
180	0\\
180	0\\
180	0\\
180	0\\
180	0\\
180	0\\
180	0\\
180	0\\
180	0\\
180	0\\
180	0\\
180	0\\
180	0\\
180	0\\
180	0\\
180	0\\
180	0\\
180	0\\
180	0\\
180	0\\
180	3.46207919166248e-24\\
180	0.291089664529579\\
180	0.815218102756897\\
179.677667217548	1.52477868253858\\
178.678435591947	2.37990055445787\\
176.709788123122	3.34594441520269\\
173.557212344351	4.39108320688454\\
169.073230934763	5.48390975732009\\
163.160508862583	6.59099434245203\\
155.852118925089	7.67428583889727\\
147.297486702545	8.68830730782735\\
137.723280312707	9.57702009882087\\
127.397369978029	10.2702769783893\\
116.599724330969	10.6797573864202\\
105.59927425757	10.6942224137447\\
94.6361890518527	10.423893234195\\
83.9160380767513	9.9556321431328\\
73.6079516077997	9.3572021070737\\
63.8451250889702	8.68082818674423\\
54.7267828593082	7.96616934103279\\
46.3210136687073	7.24279055758396\\
38.6681177920476	6.53222134223424\\
31.7842603637649	5.84963237074424\\
25.6652504754591	5.20522561196579\\
20.2903212181664	4.60535249896536\\
15.6257560759077	4.05343331363678\\
11.6282665208202	3.55068063409213\\
8.24803650559684	3.0966864583855\\
5.43138099753073	2.68988940528304\\
3.12301363188964	2.32794442994214\\
1.26791468210254	2.00799728059346\\
-0.187164272560405	1.72688898788392\\
-1.2925352025654	1.48133263154407\\
-2.09469718340566	1.26802871841653\\
-2.6357146398433	1.08372252937817\\
-2.95281914610442	0.925284307714056\\
-3.07823088358906	0.789751779975032\\
-3.03918688551579	0.674297045673422\\
-2.85814365922469	0.576223279930636\\
-2.55312893077553	0.492949547834387\\
0	0\\
0	0\\
0	0\\
0	0\\
0	0\\
0	0\\
0	0\\
0	0\\
0	0\\
0	0\\
0	0\\
0	0\\
0	0\\
0	0\\
0	0\\
0	0\\
0	0\\
0	0\\
0	0\\
0	0\\
};
\addlegendentry{Trajectory}

\end{axis}
\end{tikzpicture}%
        \caption{Calculated optimal trajectory and mountain.} 
\label{fig:mountain} 
\end{figure}


\subsection{Optimal open loop control}
The optimal control input sequence $u^*$ and optimal trajectory $x^*$ were exported to Simulink. The helicopter response under this scheme can be seen in figure \ref{fig:two_dim_open_loop}.
\begin{figure}[H] 
        \centering
        \setlength{\figureheight}{6cm}
        \setlength{\figurewidth}{10cm}
        % This file was created by matlab2tikz.
%
\begin{tikzpicture}

\begin{axis}[%
width=0.951\figurewidth,
height=0.419\figureheight,
at={(0\figurewidth,0.581\figureheight)},
scale only axis,
xmin=0,
xmax=25,
ymin=-50,
ymax=200,
ylabel={$\lambda\text{ [deg]}$},
axis background/.style={fill=white},
legend style={legend cell align=left, align=left, draw=black}
]
\addplot [color=blue]
  table[row sep=crcr]{%
0	180\\
0.25	180\\
0.5	180\\
0.75	180\\
1	180\\
1.25	180\\
1.5	180\\
1.75	180\\
2	180\\
2.25	180\\
2.5	180\\
2.75	180\\
3	180\\
3.25	180\\
3.5	180\\
3.75	180\\
4	180\\
4.25	180\\
4.5	180\\
4.75	180\\
5	180\\
5.25	180\\
5.5	180\\
5.75	180\\
6	179.68\\
6.25	178.68\\
6.5	176.71\\
6.75	173.56\\
7	169.07\\
7.25	163.16\\
7.5	155.85\\
7.75	147.3\\
8	137.72\\
8.25	127.4\\
8.5	116.6\\
8.75	105.6\\
9	94.636\\
9.25	83.916\\
9.5	73.608\\
9.75	63.845\\
10	54.727\\
10.25	46.321\\
10.5	38.668\\
10.75	31.784\\
11	25.665\\
11.25	20.29\\
11.5	15.626\\
11.75	11.628\\
12	8.248\\
12.25	5.4314\\
12.5	3.123\\
12.75	1.2679\\
13	-0.18716\\
13.25	-1.2925\\
13.5	-2.0947\\
13.75	-2.6357\\
14	-2.9528\\
14.25	-3.0782\\
14.5	-3.0392\\
14.75	-2.8581\\
15	-2.5531\\
15.25	0\\
15.5	0\\
15.75	0\\
16	0\\
16.25	0\\
16.5	0\\
16.75	0\\
17	0\\
17.25	0\\
17.5	0\\
17.75	0\\
18	0\\
18.25	0\\
18.5	0\\
18.75	0\\
19	0\\
19.25	0\\
19.5	0\\
19.75	0\\
20	0\\
};
\addlegendentry{$\lambda{}_{\text{ref}}$}

\addplot [color=black!50!green]
  table[row sep=crcr]{%
0	180\\
0.1	180.04\\
0.2	180.22\\
0.3	180.53\\
0.4	180.88\\
0.5	181.23\\
0.6	181.58\\
0.7	181.93\\
0.8	182.24\\
0.9	182.5\\
1	182.77\\
1.1	183.03\\
1.2	183.25\\
1.3	183.52\\
1.4	183.74\\
1.5	184\\
1.6	184.22\\
1.7	184.48\\
1.8	184.75\\
1.9	185.01\\
2	185.27\\
2.1	185.58\\
2.2	185.89\\
2.3	186.2\\
2.4	186.55\\
2.5	186.86\\
2.6	187.21\\
2.7	187.6\\
2.8	187.95\\
2.9	188.31\\
3	188.7\\
3.1	189.05\\
3.2	189.45\\
3.3	189.8\\
3.4	190.2\\
3.5	190.59\\
3.6	190.99\\
3.7	191.34\\
3.8	191.73\\
3.9	192.13\\
4	192.52\\
4.1	192.92\\
4.2	193.32\\
4.3	193.71\\
4.4	194.06\\
4.5	194.46\\
4.6	194.85\\
4.7	195.25\\
4.8	195.6\\
4.9	196\\
5	196.39\\
5.1	196.74\\
5.2	197.09\\
5.3	197.36\\
5.4	197.49\\
5.5	197.45\\
5.6	197.27\\
5.7	196.96\\
5.8	196.44\\
5.9	195.78\\
6	194.9\\
6.1	193.89\\
6.2	192.61\\
6.3	191.16\\
6.4	189.45\\
6.5	187.47\\
6.6	185.23\\
6.7	182.77\\
6.8	180\\
6.9	177.01\\
7	173.76\\
7.1	170.33\\
7.2	166.68\\
7.3	162.86\\
7.4	158.86\\
7.5	154.73\\
7.6	150.51\\
7.7	146.25\\
7.8	141.9\\
7.9	137.46\\
8	133.07\\
8.1	128.63\\
8.2	124.23\\
8.3	119.84\\
8.4	115.44\\
8.5	111.09\\
8.6	106.83\\
8.7	102.61\\
8.8	98.438\\
8.9	94.395\\
9	90.439\\
9.1	86.616\\
9.2	82.881\\
9.3	79.233\\
9.4	75.762\\
9.5	72.378\\
9.6	69.126\\
9.7	66.05\\
9.8	63.062\\
9.9	60.249\\
10	57.568\\
10.1	55.063\\
10.2	52.69\\
10.3	50.449\\
10.4	48.384\\
10.5	46.406\\
10.6	44.648\\
10.7	43.022\\
10.8	41.484\\
10.9	40.166\\
11	38.936\\
11.1	37.881\\
11.2	36.958\\
11.3	36.123\\
11.4	35.42\\
11.5	34.893\\
11.6	34.453\\
11.7	34.058\\
11.8	33.838\\
11.9	33.706\\
12	33.662\\
12.1	33.706\\
12.2	33.838\\
12.3	34.058\\
12.4	34.321\\
12.5	34.673\\
12.6	35.112\\
12.7	35.64\\
12.8	36.167\\
12.9	36.782\\
13	37.441\\
13.1	38.188\\
13.2	38.979\\
13.3	39.771\\
13.4	40.649\\
13.5	41.528\\
13.6	42.451\\
13.7	43.418\\
13.8	44.429\\
13.9	45.483\\
14	46.538\\
14.1	47.637\\
14.2	48.735\\
14.3	49.878\\
14.4	51.021\\
14.5	52.207\\
14.6	53.438\\
14.7	54.624\\
14.8	55.854\\
14.9	57.085\\
15	58.359\\
15.1	59.59\\
15.2	60.908\\
15.3	62.183\\
15.4	63.413\\
15.5	64.688\\
15.6	65.962\\
15.7	67.236\\
15.8	68.467\\
15.9	69.741\\
16	70.972\\
16.1	72.202\\
16.2	73.433\\
16.3	74.663\\
16.4	75.85\\
16.5	77.08\\
16.6	78.267\\
16.7	79.453\\
16.8	80.64\\
16.9	81.782\\
17	82.925\\
17.1	84.023\\
17.2	85.122\\
17.3	86.265\\
17.4	87.319\\
17.5	88.374\\
17.6	89.385\\
17.7	90.439\\
17.8	91.406\\
17.9	92.373\\
18	93.34\\
18.1	94.263\\
18.2	95.186\\
18.3	96.064\\
18.4	96.899\\
18.5	97.734\\
18.6	98.569\\
18.7	99.36\\
18.8	100.15\\
18.9	100.9\\
19	101.65\\
19.1	102.35\\
19.2	103.05\\
19.3	103.75\\
19.4	104.41\\
19.5	105.12\\
19.6	105.73\\
19.7	106.39\\
19.8	107.01\\
19.9	107.58\\
20	108.19\\
20.1	108.72\\
20.2	109.29\\
20.3	109.86\\
20.4	110.35\\
20.5	110.87\\
20.6	111.36\\
20.7	111.84\\
20.8	112.28\\
20.9	112.72\\
21	113.16\\
21.1	113.55\\
21.2	113.95\\
21.3	114.35\\
21.4	114.7\\
21.5	115.05\\
21.6	115.4\\
21.7	115.71\\
21.8	116.02\\
21.9	116.32\\
22	116.63\\
22.1	116.89\\
};
\addlegendentry{$\lambda$}

\end{axis}

\begin{axis}[%
width=0.951\figurewidth,
height=0.419\figureheight,
at={(0\figurewidth,0\figureheight)},
scale only axis,
xmin=0,
xmax=25,
xlabel={Time [s]},
ymin=-30,
ymax=20,
ylabel={e [deg]},
axis background/.style={fill=white},
legend style={legend cell align=left, align=left, draw=black}
]
\addplot [color=blue]
  table[row sep=crcr]{%
0	0\\
0.25	0\\
0.5	0\\
0.75	0\\
1	0\\
1.25	0\\
1.5	0\\
1.75	0\\
2	0\\
2.25	0\\
2.5	0\\
2.75	0\\
3	0\\
3.25	0\\
3.5	0\\
3.75	0\\
4	0\\
4.25	0\\
4.5	0\\
4.75	0\\
5	0\\
5.25	3.4621e-24\\
5.5	0.29109\\
5.75	0.81522\\
6	1.5248\\
6.25	2.3799\\
6.5	3.3459\\
6.75	4.3911\\
7	5.4839\\
7.25	6.591\\
7.5	7.6743\\
7.75	8.6883\\
8	9.577\\
8.25	10.27\\
8.5	10.68\\
8.75	10.694\\
9	10.424\\
9.25	9.9556\\
9.5	9.3572\\
9.75	8.6808\\
10	7.9662\\
10.25	7.2428\\
10.5	6.5322\\
10.75	5.8496\\
11	5.2052\\
11.25	4.6054\\
11.5	4.0534\\
11.75	3.5507\\
12	3.0967\\
12.25	2.6899\\
12.5	2.3279\\
12.75	2.008\\
13	1.7269\\
13.25	1.4813\\
13.5	1.268\\
13.75	1.0837\\
14	0.92528\\
14.25	0.78975\\
14.5	0.6743\\
14.75	0.57622\\
15	0.49295\\
15.25	0\\
15.5	0\\
15.75	0\\
16	0\\
16.25	0\\
16.5	0\\
16.75	0\\
17	0\\
17.25	0\\
17.5	0\\
17.75	0\\
18	0\\
18.25	0\\
18.5	0\\
18.75	0\\
19	0\\
19.25	0\\
19.5	0\\
19.75	0\\
20	0\\
};
\addlegendentry{$\text{e}_{\text{ref}}$}

\addplot [color=black!50!green]
  table[row sep=crcr]{%
0	-30\\
0.1	-30\\
0.2	-29.824\\
0.3	-29.561\\
0.4	-28.77\\
0.5	-27.627\\
0.6	-26.045\\
0.7	-24.287\\
0.8	-22.354\\
0.9	-20.332\\
1	-18.398\\
1.1	-16.465\\
1.2	-14.619\\
1.3	-12.949\\
1.4	-11.455\\
1.5	-10.049\\
1.6	-8.8184\\
1.7	-7.6758\\
1.8	-6.7969\\
1.9	-6.0059\\
2	-5.3906\\
2.1	-4.8633\\
2.2	-4.5117\\
2.3	-4.1602\\
2.4	-3.8965\\
2.5	-3.7207\\
2.6	-3.5449\\
2.7	-3.3691\\
2.8	-3.2812\\
2.9	-3.1055\\
3	-3.0176\\
3.1	-2.8418\\
3.2	-2.7539\\
3.3	-2.5781\\
3.4	-2.4902\\
3.5	-2.3145\\
3.6	-2.2266\\
3.7	-2.1387\\
3.8	-2.0508\\
3.9	-1.9629\\
4	-1.9629\\
4.1	-1.875\\
4.2	-1.875\\
4.3	-1.7871\\
4.4	-1.7871\\
4.5	-1.6992\\
4.6	-1.6992\\
4.7	-1.6992\\
4.8	-1.6113\\
4.9	-1.6113\\
5	-1.5234\\
5.1	-1.3477\\
5.2	-0.99609\\
5.3	-0.64453\\
5.4	-0.20508\\
5.5	0.32227\\
5.6	0.84961\\
5.7	1.377\\
5.8	1.8164\\
5.9	2.2559\\
6	2.6074\\
6.1	2.8711\\
6.2	3.1348\\
6.3	3.3105\\
6.4	3.4863\\
6.5	3.6621\\
6.6	3.75\\
6.7	3.75\\
6.8	3.8379\\
6.9	3.8379\\
7	3.9258\\
7.1	3.9258\\
7.2	3.9258\\
7.3	4.1016\\
7.4	4.1895\\
7.5	4.2773\\
7.6	4.4531\\
7.7	4.541\\
7.8	4.6289\\
7.9	4.7168\\
8	4.7168\\
8.1	4.8047\\
8.2	4.7168\\
8.3	4.541\\
8.4	4.2773\\
8.5	4.0137\\
8.6	3.6621\\
8.7	3.3105\\
8.8	2.959\\
8.9	2.6074\\
9	2.2559\\
9.1	1.9922\\
9.2	1.7285\\
9.3	1.5527\\
9.4	1.377\\
9.5	1.2012\\
9.6	1.0254\\
9.7	0.9375\\
9.8	0.76172\\
9.9	0.67383\\
10	0.58594\\
10.1	0.41016\\
10.2	0.32227\\
10.3	0.23438\\
10.4	0.14648\\
10.5	-0.029297\\
10.6	-0.029297\\
10.7	-0.11719\\
10.8	-0.20508\\
10.9	-0.29297\\
11	-0.29297\\
11.1	-0.29297\\
11.2	-0.38086\\
11.3	-0.38086\\
11.4	-0.38086\\
11.5	-0.38086\\
11.6	-0.29297\\
11.7	-0.29297\\
11.8	-0.29297\\
11.9	-0.20508\\
12	-0.20508\\
12.1	-0.11719\\
12.2	-0.11719\\
12.3	-0.11719\\
12.4	-0.029297\\
12.5	-0.029297\\
12.6	-0.029297\\
12.7	0.058594\\
12.8	0.058594\\
12.9	0.14648\\
13	0.14648\\
13.1	0.23438\\
13.2	0.32227\\
13.3	0.32227\\
13.4	0.41016\\
13.5	0.41016\\
13.6	0.49805\\
13.7	0.58594\\
13.8	0.58594\\
13.9	0.67383\\
14	0.67383\\
14.1	0.76172\\
14.2	0.76172\\
14.3	0.76172\\
14.4	0.76172\\
14.5	0.76172\\
14.6	0.84961\\
14.7	0.84961\\
14.8	0.84961\\
14.9	0.84961\\
15	0.84961\\
15.1	0.84961\\
15.2	0.84961\\
15.3	0.84961\\
15.4	0.84961\\
15.5	0.84961\\
15.6	0.84961\\
15.7	0.84961\\
15.8	0.84961\\
15.9	0.84961\\
16	0.84961\\
16.1	0.84961\\
16.2	0.84961\\
16.3	0.84961\\
16.4	0.84961\\
16.5	0.76172\\
16.6	0.76172\\
16.7	0.76172\\
16.8	0.84961\\
16.9	0.84961\\
17	0.84961\\
17.1	0.84961\\
17.2	0.76172\\
17.3	0.76172\\
17.4	0.76172\\
17.5	0.76172\\
17.6	0.76172\\
17.7	0.76172\\
17.8	0.76172\\
17.9	0.76172\\
18	0.76172\\
18.1	0.76172\\
18.2	0.76172\\
18.3	0.84961\\
18.4	0.84961\\
18.5	0.76172\\
18.6	0.76172\\
18.7	0.76172\\
18.8	0.76172\\
18.9	0.76172\\
19	0.76172\\
19.1	0.84961\\
19.2	0.84961\\
19.3	0.76172\\
19.4	0.76172\\
19.5	0.76172\\
19.6	0.76172\\
19.7	0.76172\\
19.8	0.67383\\
19.9	0.67383\\
20	0.67383\\
20.1	0.67383\\
20.2	0.58594\\
20.3	0.58594\\
20.4	0.58594\\
20.5	0.58594\\
20.6	0.49805\\
20.7	0.49805\\
20.8	0.49805\\
20.9	0.41016\\
21	0.41016\\
21.1	0.41016\\
21.2	0.49805\\
21.3	0.49805\\
21.4	0.49805\\
21.5	0.58594\\
21.6	0.58594\\
21.7	0.67383\\
21.8	0.67383\\
21.9	0.76172\\
22	0.76172\\
22.1	0.76172\\
};
\addlegendentry{e}

\end{axis}
\end{tikzpicture}%
        \caption{Optimal open loop control under mountain constraint.} 
\label{fig:two_dim_open_loop} 
\end{figure}
As we saw in \cref{mountain}, the elevation reference, $e_{ref}$ barley goes over the mountain, thus since the elevation goes below the elevation reference, the helicopter crashes into the mountain. Also the travel drifts considerably even in the beginning of the experiment. We clearly need some feedback to deal with the travel drifting, but the PID elevation controller should make the elevation keep its reference. However there is a problem, because the reference is not a constant reference, but more like a ramp, thus one integrator in the regulator is not enough to prevent some constant deviation. Also there is a problem in the model witch we will discuss in \cref{sec:decouple_discussion}.  

\subsection{Optimal closed loop control}
To introduce feedback, we used the state- and drive penalties defined in equation \ref{eq:two_dim_lqr_matrices}. The MATLAB code for this can be seen in listing \ref{lst:two_dim_lqr_K}.
\begin{equation}\label{eq:two_dim_lqr_matrices}
    Q = \begin{bmatrix}
        100 & 0 & 0 & 0 & 0 & 0\\
        0 & 0 & 0 & 0 & 0 & 0\\
        0 & 0 & 0 & 0 & 0 & 0\\
        0 & 0 & 0 & 0 & 0 & 0\\
        0 & 0 & 0 & 0 & 100 & 0\\
        0 & 0 & 0 & 0 & 0 & 0
    \end{bmatrix}\qquad
    R = \begin{bmatrix}
        1 & 0\\
        0 & 1
    \end{bmatrix}
\end{equation}
\begin{lstlisting}[caption=MATLAB code to generate LQR feedback.,label=lst:two_dim_lqr_K]
%% Generate K for feedback under optimal control
R = diag([1, 1]);
Q_feedback = diag([100, 0,0,0,100,0]);
[K, P, e] = dlqr(A, B, Q_feedback, R, []);
\end{lstlisting}
With feedback, the situation improves a little, which can be seen in figure \ref{fig:two_dim_closed_loop}. Now, the helicopter definitely swipes out a satisfactory trajectory as far as travel is concerned. However, it still does not make it "over the mountain". This seems to be a constant - no matter how heavily deviations in elevation was penalized, the helicopter would never make it over - the only result would be deteriorating travel fidelities. We believe this occurs because elevation and elevation rate is decoupled from the other states in our model, as discussed in section \ref{sec:decouple_discussion}.
\begin{figure} 
        \centering
        \setlength{\figureheight}{6cm}
        \setlength{\figurewidth}{10cm}
        % This file was created by matlab2tikz.
%
\begin{tikzpicture}

\begin{axis}[%
width=0.951\figurewidth,
height=0.419\figureheight,
at={(0\figurewidth,0.581\figureheight)},
scale only axis,
xmin=0,
xmax=25,
ymin=-50,
ymax=200,
ylabel={e [deg]},
axis background/.style={fill=white},
legend style={legend cell align=left, align=left, draw=black}
]
\addplot [color=blue]
  table[row sep=crcr]{%
0	180\\
0.25	180\\
0.5	180\\
0.75	180\\
1	180\\
1.25	180\\
1.5	180\\
1.75	180\\
2	180\\
2.25	180\\
2.5	180\\
2.75	180\\
3	180\\
3.25	180\\
3.5	180\\
3.75	180\\
4	180\\
4.25	180\\
4.5	180\\
4.75	180\\
5	180\\
5.25	180\\
5.5	180\\
5.75	180\\
6	179.68\\
6.25	178.68\\
6.5	176.71\\
6.75	173.56\\
7	169.07\\
7.25	163.16\\
7.5	155.85\\
7.75	147.3\\
8	137.72\\
8.25	127.4\\
8.5	116.6\\
8.75	105.6\\
9	94.636\\
9.25	83.916\\
9.5	73.608\\
9.75	63.845\\
10	54.727\\
10.25	46.321\\
10.5	38.668\\
10.75	31.784\\
11	25.665\\
11.25	20.29\\
11.5	15.626\\
11.75	11.628\\
12	8.248\\
12.25	5.4314\\
12.5	3.123\\
12.75	1.2679\\
13	-0.18716\\
13.25	-1.2925\\
13.5	-2.0947\\
13.75	-2.6357\\
14	-2.9528\\
14.25	-3.0782\\
14.5	-3.0392\\
14.75	-2.8581\\
15	-2.5531\\
15.25	0\\
15.5	0\\
15.75	0\\
16	0\\
16.25	0\\
16.5	0\\
16.75	0\\
17	0\\
17.25	0\\
17.5	0\\
17.75	0\\
18	0\\
18.25	0\\
18.5	0\\
18.75	0\\
19	0\\
19.25	0\\
19.5	0\\
19.75	0\\
20	0\\
};
\addlegendentry{$\text{e}_{\text{ref}}$}

\addplot [color=black!50!green]
  table[row sep=crcr]{%
0	180\\
0.1	180.04\\
0.2	180.22\\
0.3	180.48\\
0.4	180.92\\
0.5	181.32\\
0.6	181.8\\
0.7	182.11\\
0.8	182.42\\
0.9	182.64\\
1	182.81\\
1.1	182.99\\
1.2	183.08\\
1.3	183.21\\
1.4	183.34\\
1.5	183.43\\
1.6	183.56\\
1.7	183.74\\
1.8	183.87\\
1.9	184.04\\
2	184.22\\
2.1	184.35\\
2.2	184.53\\
2.3	184.7\\
2.4	184.83\\
2.5	184.97\\
2.6	185.05\\
2.7	185.14\\
2.8	185.23\\
2.9	185.27\\
3	185.32\\
3.1	185.32\\
3.2	185.32\\
3.3	185.32\\
3.4	185.32\\
3.5	185.32\\
3.6	185.27\\
3.7	185.23\\
3.8	185.23\\
3.9	185.19\\
4	185.14\\
4.1	185.1\\
4.2	185.05\\
4.3	185.05\\
4.4	185.01\\
4.5	184.97\\
4.6	184.97\\
4.7	184.92\\
4.8	184.92\\
4.9	184.88\\
5	184.88\\
5.1	184.83\\
5.2	184.83\\
5.3	184.75\\
5.4	184.53\\
5.5	184.26\\
5.6	183.91\\
5.7	183.43\\
5.8	182.86\\
5.9	182.2\\
6	181.36\\
6.1	180.4\\
6.2	179.3\\
6.3	177.98\\
6.4	176.48\\
6.5	174.81\\
6.6	172.92\\
6.7	170.86\\
6.8	168.57\\
6.9	166.11\\
7	163.48\\
7.1	160.71\\
7.2	157.72\\
7.3	154.64\\
7.4	151.39\\
7.5	148.05\\
7.6	144.67\\
7.7	141.2\\
7.8	137.68\\
7.9	134.12\\
8	130.52\\
8.1	126.96\\
8.2	123.35\\
8.3	119.75\\
8.4	116.19\\
8.5	112.63\\
8.6	109.12\\
8.7	105.64\\
8.8	102.17\\
8.9	98.745\\
9	95.361\\
9.1	92.021\\
9.2	88.77\\
9.3	85.518\\
9.4	82.354\\
9.5	79.233\\
9.6	76.157\\
9.7	73.169\\
9.8	70.269\\
9.9	67.412\\
10	64.644\\
10.1	61.919\\
10.2	59.282\\
10.3	56.689\\
10.4	54.229\\
10.5	51.812\\
10.6	49.482\\
10.7	47.197\\
10.8	45\\
10.9	42.891\\
11	40.869\\
11.1	38.936\\
11.2	37.002\\
11.3	35.244\\
11.4	33.486\\
11.5	31.86\\
11.6	30.278\\
11.7	28.784\\
11.8	27.334\\
11.9	26.016\\
12	24.741\\
12.1	23.555\\
12.2	22.412\\
12.3	21.357\\
12.4	20.347\\
12.5	19.38\\
12.6	18.457\\
12.7	17.622\\
12.8	16.787\\
12.9	15.996\\
13	15.293\\
13.1	14.59\\
13.2	13.931\\
13.3	13.315\\
13.4	12.744\\
13.5	12.217\\
13.6	11.689\\
13.7	11.25\\
13.8	10.811\\
13.9	10.415\\
14	10.02\\
14.1	9.7119\\
14.2	9.4043\\
14.3	9.1406\\
14.4	8.877\\
14.5	8.6572\\
14.6	8.4375\\
14.7	8.3057\\
14.8	8.1299\\
14.9	8.042\\
15	7.9541\\
15.1	7.8662\\
15.2	7.8223\\
15.3	7.8223\\
15.4	7.8223\\
15.5	7.8223\\
15.6	7.8223\\
15.7	7.8223\\
15.8	7.8223\\
15.9	7.8223\\
16	7.8223\\
16.1	7.7783\\
16.2	7.7783\\
16.3	7.7344\\
16.4	7.6904\\
16.5	7.6465\\
16.6	7.5586\\
16.7	7.5146\\
16.8	7.4268\\
16.9	7.3389\\
17	7.251\\
17.1	7.1631\\
17.2	7.0752\\
17.3	6.9873\\
17.4	6.8555\\
17.5	6.7676\\
17.6	6.6797\\
17.7	6.5918\\
17.8	6.5039\\
17.9	6.416\\
18	6.3281\\
18.1	6.2402\\
18.2	6.1963\\
18.3	6.1084\\
18.4	6.0645\\
18.5	6.0205\\
18.6	5.9766\\
18.7	5.9326\\
18.8	5.8887\\
18.9	5.8447\\
19	5.8447\\
19.1	5.8008\\
19.2	5.8008\\
19.3	5.7568\\
19.4	5.7568\\
19.5	5.7568\\
19.6	5.7129\\
19.7	5.7129\\
19.8	5.7129\\
19.9	5.6689\\
20	5.6689\\
20.1	5.6689\\
20.2	5.6689\\
20.3	5.6689\\
20.4	5.6689\\
20.5	5.625\\
20.6	5.625\\
20.7	5.625\\
20.8	5.625\\
20.9	5.5811\\
21	5.5811\\
21.1	5.5371\\
21.2	5.5371\\
21.3	5.4932\\
21.4	5.4932\\
21.5	5.4492\\
};
\addlegendentry{e}

\end{axis}

\begin{axis}[%
width=0.951\figurewidth,
height=0.419\figureheight,
at={(0\figurewidth,0\figureheight)},
scale only axis,
xmin=0,
xmax=25,
xlabel={Time [s]},
ymin=-30,
ymax=20,
ylabel={$\lambda\text{ [deg]}$},
axis background/.style={fill=white},
legend style={legend cell align=left, align=left, draw=black}
]
\addplot [color=blue]
  table[row sep=crcr]{%
0	0\\
0.25	0\\
0.5	0\\
0.75	0\\
1	0\\
1.25	0\\
1.5	0\\
1.75	0\\
2	0\\
2.25	0\\
2.5	0\\
2.75	0\\
3	0\\
3.25	0\\
3.5	0\\
3.75	0\\
4	0\\
4.25	0\\
4.5	0\\
4.75	0\\
5	0\\
5.25	3.4621e-24\\
5.5	0.29109\\
5.75	0.81522\\
6	1.5248\\
6.25	2.3799\\
6.5	3.3459\\
6.75	4.3911\\
7	5.4839\\
7.25	6.591\\
7.5	7.6743\\
7.75	8.6883\\
8	9.577\\
8.25	10.27\\
8.5	10.68\\
8.75	10.694\\
9	10.424\\
9.25	9.9556\\
9.5	9.3572\\
9.75	8.6808\\
10	7.9662\\
10.25	7.2428\\
10.5	6.5322\\
10.75	5.8496\\
11	5.2052\\
11.25	4.6054\\
11.5	4.0534\\
11.75	3.5507\\
12	3.0967\\
12.25	2.6899\\
12.5	2.3279\\
12.75	2.008\\
13	1.7269\\
13.25	1.4813\\
13.5	1.268\\
13.75	1.0837\\
14	0.92528\\
14.25	0.78975\\
14.5	0.6743\\
14.75	0.57622\\
15	0.49295\\
15.25	0\\
15.5	0\\
15.75	0\\
16	0\\
16.25	0\\
16.5	0\\
16.75	0\\
17	0\\
17.25	0\\
17.5	0\\
17.75	0\\
18	0\\
18.25	0\\
18.5	0\\
18.75	0\\
19	0\\
19.25	0\\
19.5	0\\
19.75	0\\
20	0\\
};
\addlegendentry{$\lambda{}_{\text{ref}}$}

\addplot [color=black!50!green]
  table[row sep=crcr]{%
0	-30\\
0.1	-30\\
0.2	-29.912\\
0.3	-29.561\\
0.4	-28.682\\
0.5	-27.188\\
0.6	-25.078\\
0.7	-22.617\\
0.8	-19.98\\
0.9	-17.432\\
1	-14.971\\
1.1	-12.773\\
1.2	-10.752\\
1.3	-8.9941\\
1.4	-7.4121\\
1.5	-6.1816\\
1.6	-5.2148\\
1.7	-4.5117\\
1.8	-3.9844\\
1.9	-3.6328\\
2	-3.3691\\
2.1	-3.2812\\
2.2	-3.1934\\
2.3	-3.0176\\
2.4	-2.9297\\
2.5	-2.7539\\
2.6	-2.4902\\
2.7	-2.3145\\
2.8	-2.0508\\
2.9	-1.875\\
3	-1.6992\\
3.1	-1.5234\\
3.2	-1.4355\\
3.3	-1.3477\\
3.4	-1.2598\\
3.5	-1.1719\\
3.6	-1.084\\
3.7	-0.99609\\
3.8	-0.99609\\
3.9	-0.9082\\
4	-0.9082\\
4.1	-0.82031\\
4.2	-0.73242\\
4.3	-0.73242\\
4.4	-0.73242\\
4.5	-0.64453\\
4.6	-0.64453\\
4.7	-0.64453\\
4.8	-0.55664\\
4.9	-0.55664\\
5	-0.46875\\
5.1	-0.29297\\
5.2	-0.029297\\
5.3	0.32227\\
5.4	0.67383\\
5.5	1.2012\\
5.6	1.5527\\
5.7	1.9922\\
5.8	2.3438\\
5.9	2.6953\\
6	2.959\\
6.1	3.2227\\
6.2	3.4863\\
6.3	3.6621\\
6.4	3.9258\\
6.5	4.1895\\
6.6	4.3652\\
6.7	4.6289\\
6.8	4.8047\\
6.9	4.9805\\
7	5.2441\\
7.1	5.4199\\
7.2	5.6836\\
7.3	5.8594\\
7.4	6.123\\
7.5	6.2988\\
7.6	6.5625\\
7.7	6.7383\\
7.8	6.8262\\
7.9	6.9141\\
8	7.002\\
8.1	7.002\\
8.2	6.9141\\
8.3	6.8262\\
8.4	6.6504\\
8.5	6.3867\\
8.6	6.123\\
8.7	5.8594\\
8.8	5.6836\\
8.9	5.4199\\
9	5.2441\\
9.1	5.0684\\
9.2	4.9805\\
9.3	4.8926\\
9.4	4.8047\\
9.5	4.7168\\
9.6	4.6289\\
9.7	4.541\\
9.8	4.3652\\
9.9	4.2773\\
10	4.1016\\
10.1	4.0137\\
10.2	3.8379\\
10.3	3.75\\
10.4	3.6621\\
10.5	3.5742\\
10.6	3.5742\\
10.7	3.4863\\
10.8	3.3984\\
10.9	3.3105\\
11	3.2227\\
11.1	3.1348\\
11.2	3.0469\\
11.3	2.959\\
11.4	2.959\\
11.5	2.8711\\
11.6	2.7832\\
11.7	2.6953\\
11.8	2.6074\\
11.9	2.5195\\
12	2.4316\\
12.1	2.3438\\
12.2	2.3438\\
12.3	2.2559\\
12.4	2.168\\
12.5	2.0801\\
12.6	2.0801\\
12.7	1.9922\\
12.8	1.9922\\
12.9	1.9043\\
13	1.9043\\
13.1	1.8164\\
13.2	1.8164\\
13.3	1.8164\\
13.4	1.7285\\
13.5	1.7285\\
13.6	1.7285\\
13.7	1.7285\\
13.8	1.6406\\
13.9	1.6406\\
14	1.5527\\
14.1	1.5527\\
14.2	1.5527\\
14.3	1.4648\\
14.4	1.4648\\
14.5	1.4648\\
14.6	1.4648\\
14.7	1.4648\\
14.8	1.4648\\
14.9	1.4648\\
15	1.377\\
15.1	1.377\\
15.2	1.377\\
15.3	1.377\\
15.4	1.377\\
15.5	1.377\\
15.6	1.2891\\
15.7	1.2891\\
15.8	1.2891\\
15.9	1.2891\\
16	1.2012\\
16.1	1.2012\\
16.2	1.2012\\
16.3	1.1133\\
16.4	1.1133\\
16.5	1.1133\\
16.6	1.1133\\
16.7	1.0254\\
16.8	1.0254\\
16.9	1.0254\\
17	1.0254\\
17.1	1.0254\\
17.2	0.9375\\
17.3	0.9375\\
17.4	0.9375\\
17.5	0.84961\\
17.6	0.84961\\
17.7	0.84961\\
17.8	0.84961\\
17.9	0.84961\\
18	0.84961\\
18.1	0.76172\\
18.2	0.76172\\
18.3	0.76172\\
18.4	0.76172\\
18.5	0.67383\\
18.6	0.67383\\
18.7	0.67383\\
18.8	0.67383\\
18.9	0.67383\\
19	0.67383\\
19.1	0.67383\\
19.2	0.67383\\
19.3	0.67383\\
19.4	0.67383\\
19.5	0.67383\\
19.6	0.67383\\
19.7	0.67383\\
19.8	0.67383\\
19.9	0.67383\\
20	0.67383\\
20.1	0.67383\\
20.2	0.67383\\
20.3	0.76172\\
20.4	0.76172\\
20.5	0.76172\\
20.6	0.67383\\
20.7	0.67383\\
20.8	0.67383\\
20.9	0.67383\\
21	0.67383\\
21.1	0.67383\\
21.2	0.67383\\
21.3	0.67383\\
21.4	0.58594\\
21.5	0.58594\\
};
\addlegendentry{$\lambda$}

\end{axis}
\end{tikzpicture}%
        \caption{Optimal closed loop control under mountain constraint.} 
\label{fig:two_dim_closed_loop} 
\end{figure}
\subsection{Decoupled elevation and elevation rate}\label{sec:decouple_discussion}
As can be seen in equation \ref{eq:two_dim_A_c}, elevation and elevation rate does not in any way depend on pitch, according to our model. This is clearly wrong, as any pitch angle other than neutral will reduce the upward lift being generated. In fact, we have that $F_u = F_s\cdot\cos\theta$, where $F_u$ is the upward force, $F_s$ is the motor sum as before, and $\theta$ is the angular deviation from neutral in the pitch.\\
\\
Therefore, our model is completely oblivious to the fact that changing the pitch will reduce lift. This could explain why the helicopter is unable to fly over the mountain, even in the case of feedback. One solution would be to move into the strange and unfamiliar lands of non-linear models, acknowledging the fact that changing pitch reduces the elevation rate.

\section{Optional constraints}\label{sec:optional_constraints}
In an attempt to compensate for modelling inaccuracies, and "force" the helicopter over the mountain, a further restriction of the allowed pitch was implemented. The maximum bounds for pitch was reduced from $-35^\circ\leq p\leq 35^\circ$ to $-20^\circ\leq p\leq 20^\circ$. This was the only change necessary to provoke the much better response seen in figure \ref{fig:closed_loop_restricted}.\\
\\
As is seen, the helicopter still does not quite make it past the tip of the mountain. This is however, the best behaviour we could provoke. Any further restrictions to $p$ or $\dot{p}$ tended to reduce the travel fidelity.
\begin{figure}
        \centering
        \setlength{\figureheight}{6cm}
        \setlength{\figurewidth}{10cm}
        % This file was created by matlab2tikz.
%
\begin{tikzpicture}

\begin{axis}[%
width=0.951\figurewidth,
height=0.419\figureheight,
at={(0\figurewidth,0.581\figureheight)},
scale only axis,
xmin=0,
xmax=20,
ymin=0,
ymax=200,
ylabel={$\lambda\text{ [deg]}$},
axis background/.style={fill=white},
legend style={legend cell align=left, align=left, draw=black}
]
\addplot [color=blue]
  table[row sep=crcr]{%
0	180\\
0.25	180\\
0.5	180\\
0.75	180\\
1	180\\
1.25	180\\
1.5	180\\
1.75	180\\
2	180\\
2.25	180\\
2.5	180\\
2.75	180\\
3	180\\
3.25	180\\
3.5	180\\
3.75	180\\
4	180\\
4.25	180\\
4.5	180\\
4.75	180\\
5	180\\
5.25	180\\
5.5	180\\
5.75	180\\
6	179.86\\
6.25	179.41\\
6.5	178.54\\
6.75	177.14\\
7	175.14\\
7.25	172.52\\
7.5	169.23\\
7.75	165.25\\
8	160.59\\
8.25	155.23\\
8.5	149.17\\
8.75	142.41\\
9	134.94\\
9.25	126.8\\
9.5	118.1\\
9.75	108.98\\
10	99.598\\
10.25	90.122\\
10.5	80.709\\
10.75	71.501\\
11	62.625\\
11.25	54.185\\
11.5	46.269\\
11.75	38.939\\
12	32.24\\
12.25	26.199\\
12.5	20.834\\
12.75	16.148\\
13	12.135\\
13.25	8.7743\\
13.5	6.0397\\
13.75	3.8942\\
14	2.2918\\
14.25	1.1765\\
14.5	0.48024\\
14.75	0.12106\\
15	2.0911e-18\\
15.25	0\\
15.5	0\\
15.75	0\\
16	0\\
16.25	0\\
16.5	0\\
16.75	0\\
17	0\\
17.25	0\\
17.5	0\\
17.75	0\\
18	0\\
18.25	0\\
18.5	0\\
18.75	0\\
19	0\\
19.25	0\\
19.5	0\\
19.75	0\\
20	0\\
};
\addlegendentry{$\lambda{}_{\text{ref}}$}

\addplot [color=black!50!green]
  table[row sep=crcr]{%
0	180\\
0.1	180.09\\
0.2	180.22\\
0.3	180.53\\
0.4	180.88\\
0.5	181.27\\
0.6	181.98\\
0.7	182.46\\
0.8	182.9\\
0.9	183.25\\
1	183.38\\
1.1	183.43\\
1.2	183.38\\
1.3	183.25\\
1.4	183.03\\
1.5	182.81\\
1.6	182.55\\
1.7	182.42\\
1.8	182.37\\
1.9	182.37\\
2	182.5\\
2.1	182.64\\
2.2	182.81\\
2.3	182.94\\
2.4	183.12\\
2.5	183.38\\
2.6	183.74\\
2.7	184.22\\
2.8	184.92\\
2.9	185.8\\
3	186.86\\
3.1	187.82\\
3.2	188.75\\
3.3	189.62\\
3.4	190.42\\
3.5	191.21\\
3.6	191.95\\
3.7	192.66\\
3.8	193.32\\
3.9	193.8\\
4	194.11\\
4.1	194.28\\
4.2	194.24\\
4.3	194.15\\
4.4	193.97\\
4.5	193.8\\
4.6	193.58\\
4.7	193.32\\
4.8	193.01\\
4.9	192.61\\
5	192.22\\
5.1	191.78\\
5.2	191.25\\
5.3	190.77\\
5.4	190.28\\
5.5	189.8\\
5.6	189.27\\
5.7	188.7\\
5.8	188.09\\
5.9	187.43\\
6	186.77\\
6.1	186.06\\
6.2	185.32\\
6.3	184.57\\
6.4	183.69\\
6.5	182.81\\
6.6	181.85\\
6.7	180.79\\
6.8	179.65\\
6.9	178.46\\
7	177.19\\
7.1	175.83\\
7.2	174.42\\
7.3	172.88\\
7.4	171.3\\
7.5	169.58\\
7.6	167.74\\
7.7	165.89\\
7.8	163.92\\
7.9	161.81\\
8	159.65\\
8.1	157.37\\
8.2	155\\
8.3	152.49\\
8.4	149.94\\
8.5	147.22\\
8.6	144.45\\
8.7	141.59\\
8.8	138.65\\
8.9	135.62\\
9	132.5\\
9.1	129.33\\
9.2	126.12\\
9.3	122.87\\
9.4	119.58\\
9.5	116.28\\
9.6	112.98\\
9.7	109.64\\
9.8	106.3\\
9.9	103.01\\
10	99.712\\
10.1	96.416\\
10.2	93.164\\
10.3	89.956\\
10.4	86.792\\
10.5	83.672\\
10.6	80.596\\
10.7	77.607\\
10.8	74.663\\
10.9	71.807\\
11	68.994\\
11.1	66.27\\
11.2	63.589\\
11.3	60.996\\
11.4	58.491\\
11.5	56.03\\
11.6	53.657\\
11.7	51.372\\
11.8	49.175\\
11.9	47.021\\
12	44.956\\
12.1	42.979\\
12.2	41.089\\
12.3	39.243\\
12.4	37.485\\
12.5	35.771\\
12.6	34.146\\
12.7	32.607\\
12.8	31.157\\
12.9	29.707\\
13	28.389\\
13.1	27.114\\
13.2	25.928\\
13.3	24.829\\
13.4	23.774\\
13.5	22.764\\
13.6	21.841\\
13.7	21.006\\
13.8	20.215\\
13.9	19.468\\
14	18.765\\
14.1	18.149\\
14.2	17.534\\
14.3	17.007\\
14.4	16.479\\
14.5	15.996\\
14.6	15.557\\
14.7	15.117\\
14.8	14.722\\
14.9	14.326\\
15	13.931\\
15.1	13.535\\
15.2	13.228\\
15.3	12.876\\
15.4	12.568\\
15.5	12.261\\
15.6	11.953\\
15.7	11.602\\
15.8	11.338\\
15.9	11.03\\
16	10.811\\
16.1	10.547\\
16.2	10.327\\
16.3	10.063\\
16.4	9.8438\\
16.5	9.624\\
16.6	9.4043\\
16.7	9.2285\\
16.8	9.0527\\
16.9	8.9209\\
17	8.7451\\
17.1	8.6133\\
17.2	8.4814\\
17.3	8.3936\\
17.4	8.2617\\
17.5	8.1738\\
17.6	8.0859\\
17.7	8.042\\
17.8	7.9541\\
17.9	7.9102\\
18	7.8223\\
18.1	7.7783\\
18.2	7.6904\\
18.3	7.6465\\
18.4	7.6025\\
18.5	7.5586\\
18.6	7.5146\\
18.7	7.4707\\
18.8	7.4268\\
18.9	7.3828\\
19	7.2949\\
19.1	7.251\\
19.2	7.207\\
19.3	7.1631\\
19.4	7.0752\\
19.5	7.0752\\
19.6	7.0312\\
19.7	6.9873\\
19.8	6.8994\\
};
\addlegendentry{$\lambda$}

\end{axis}

\begin{axis}[%
width=0.951\figurewidth,
height=0.419\figureheight,
at={(0\figurewidth,0\figureheight)},
scale only axis,
xmin=0,
xmax=20,
xlabel={Time [s]},
ymin=-30,
ymax=20,
ylabel={e [deg]},
axis background/.style={fill=white},
legend style={legend cell align=left, align=left, draw=black}
]
\addplot [color=blue]
  table[row sep=crcr]{%
0	0\\
0.25	0\\
0.5	0\\
0.75	0\\
1	0\\
1.25	0\\
1.5	0\\
1.75	0\\
2	0\\
2.25	0\\
2.5	0\\
2.75	0\\
3	0\\
3.25	0\\
3.5	0\\
3.75	0\\
4	0\\
4.25	0\\
4.5	0\\
4.75	0\\
5	0\\
5.25	0\\
5.5	0.17438\\
5.75	0.49331\\
6	0.93346\\
6.25	1.4766\\
6.5	2.1082\\
6.75	2.8166\\
7	3.5915\\
7.25	4.423\\
7.5	5.3004\\
7.75	6.2108\\
8	7.1378\\
8.25	8.0597\\
8.5	8.948\\
8.75	9.7648\\
9	10.46\\
9.25	10.97\\
9.5	11.21\\
9.75	11.075\\
10	10.671\\
10.25	10.082\\
10.5	9.3722\\
10.75	8.5921\\
11	7.7789\\
11.25	6.9603\\
11.5	6.1563\\
11.75	5.381\\
12	4.6439\\
12.25	3.9512\\
12.5	3.3065\\
12.75	2.7119\\
13	2.1687\\
13.25	1.6779\\
13.5	1.2409\\
13.75	0.85993\\
14	0.53852\\
14.25	0.28222\\
14.5	0.099024\\
14.75	8.3292e-20\\
15	8.6751e-20\\
15.25	0\\
15.5	0\\
15.75	0\\
16	0\\
16.25	0\\
16.5	0\\
16.75	0\\
17	0\\
17.25	0\\
17.5	0\\
17.75	0\\
18	0\\
18.25	0\\
18.5	0\\
18.75	0\\
19	0\\
19.25	0\\
19.5	0\\
19.75	0\\
20	0\\
};
\addlegendentry{$\text{e}_{\text{ref}}$}

\addplot [color=black!50!green]
  table[row sep=crcr]{%
0	-30\\
0.1	-30\\
0.2	-29.912\\
0.3	-29.561\\
0.4	-28.682\\
0.5	-27.188\\
0.6	-24.639\\
0.7	-21.299\\
0.8	-17.344\\
0.9	-13.213\\
1	-9.3457\\
1.1	-5.8301\\
1.2	-2.7539\\
1.3	-0.029297\\
1.4	1.9922\\
1.5	3.4863\\
1.6	4.2773\\
1.7	4.541\\
1.8	4.2773\\
1.9	3.4863\\
2	2.4316\\
2.1	1.1133\\
2.2	-0.20508\\
2.3	-1.3477\\
2.4	-2.3145\\
2.5	-2.9297\\
2.6	-3.3691\\
2.7	-3.5449\\
2.8	-3.457\\
2.9	-3.1055\\
3	-2.4023\\
3.1	-1.5234\\
3.2	-0.64453\\
3.3	-0.11719\\
3.4	0.23438\\
3.5	0.32227\\
3.6	0.14648\\
3.7	-0.11719\\
3.8	-0.29297\\
3.9	-0.29297\\
4	-0.20508\\
4.1	0.058594\\
4.2	0.41016\\
4.3	0.58594\\
4.4	0.58594\\
4.5	0.41016\\
4.6	0.23438\\
4.7	0.14648\\
4.8	0.14648\\
4.9	0.23438\\
5	0.49805\\
5.1	0.67383\\
5.2	0.84961\\
5.3	0.84961\\
5.4	0.84961\\
5.5	0.84961\\
5.6	1.0254\\
5.7	1.2012\\
5.8	1.4648\\
5.9	1.7285\\
6	1.9043\\
6.1	2.0801\\
6.2	2.2559\\
6.3	2.3438\\
6.4	2.6074\\
6.5	2.8711\\
6.6	3.1348\\
6.7	3.4863\\
6.8	3.75\\
6.9	4.0137\\
7	4.2773\\
7.1	4.4531\\
7.2	4.7168\\
7.3	5.0684\\
7.4	5.4199\\
7.5	5.7715\\
7.6	6.123\\
7.7	6.4746\\
7.8	6.7383\\
7.9	7.002\\
8	7.3535\\
8.1	7.6172\\
8.2	7.9688\\
8.3	8.2324\\
8.4	8.584\\
8.5	8.8477\\
8.6	9.1113\\
8.7	9.2871\\
8.8	9.5508\\
8.9	9.7266\\
9	9.9023\\
9.1	9.9902\\
9.2	10.078\\
9.3	10.078\\
9.4	10.078\\
9.5	10.078\\
9.6	9.9902\\
9.7	9.8145\\
9.8	9.7266\\
9.9	9.6387\\
10	9.4629\\
10.1	9.2871\\
10.2	9.1992\\
10.3	8.9355\\
10.4	8.7598\\
10.5	8.4961\\
10.6	8.3203\\
10.7	8.0566\\
10.8	7.793\\
10.9	7.6172\\
11	7.3535\\
11.1	7.0898\\
11.2	6.8262\\
11.3	6.5625\\
11.4	6.2988\\
11.5	6.0352\\
11.6	5.7715\\
11.7	5.5078\\
11.8	5.2441\\
11.9	4.9805\\
12	4.7168\\
12.1	4.4531\\
12.2	4.2773\\
12.3	4.1016\\
12.4	3.8379\\
12.5	3.6621\\
12.6	3.3984\\
12.7	3.2227\\
12.8	2.959\\
12.9	2.7832\\
13	2.6074\\
13.1	2.4316\\
13.2	2.2559\\
13.3	2.168\\
13.4	1.9922\\
13.5	1.8164\\
13.6	1.6406\\
13.7	1.4648\\
13.8	1.377\\
13.9	1.2891\\
14	1.2012\\
14.1	1.1133\\
14.2	1.0254\\
14.3	0.9375\\
14.4	0.9375\\
14.5	0.84961\\
14.6	0.84961\\
14.7	0.84961\\
14.8	0.76172\\
14.9	0.76172\\
15	0.67383\\
15.1	0.58594\\
15.2	0.58594\\
15.3	0.58594\\
15.4	0.58594\\
15.5	0.58594\\
15.6	0.58594\\
15.7	0.58594\\
15.8	0.49805\\
15.9	0.49805\\
16	0.41016\\
16.1	0.41016\\
16.2	0.49805\\
16.3	0.49805\\
16.4	0.49805\\
16.5	0.49805\\
16.6	0.41016\\
16.7	0.41016\\
16.8	0.32227\\
16.9	0.32227\\
17	0.41016\\
17.1	0.41016\\
17.2	0.41016\\
17.3	0.49805\\
17.4	0.41016\\
17.5	0.32227\\
17.6	0.32227\\
17.7	0.32227\\
17.8	0.32227\\
17.9	0.41016\\
18	0.41016\\
18.1	0.41016\\
18.2	0.41016\\
18.3	0.41016\\
18.4	0.32227\\
18.5	0.32227\\
18.6	0.23438\\
18.7	0.32227\\
18.8	0.32227\\
18.9	0.41016\\
19	0.41016\\
19.1	0.32227\\
19.2	0.32227\\
19.3	0.23438\\
19.4	0.23438\\
19.5	0.23438\\
19.6	0.32227\\
19.7	0.32227\\
19.8	0.32227\\
};
\addlegendentry{e}

\end{axis}
\end{tikzpicture}%
        \caption{Closed loop optimal control with $-20^\circ\leq p\leq 20^\circ$.} 
\label{fig:closed_loop_restricted} 
\end{figure}